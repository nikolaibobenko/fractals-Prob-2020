\documentclass[a4paper,11pt]{article}
\usepackage[french]{babel}
\usepackage[latin1]{inputenc}
%\usepackage{umlaut,amssymb,amsmath,amscd,a4,amsfonts}
\usepackage{amssymb,amsmath,amscd,a4,amsfonts,amsthm,mathrsfs}
%(a4 = 210 X 297 mm)
\hoffset -1in \voffset -1in \oddsidemargin 20mm \evensidemargin
\oddsidemargin \textwidth 170mm \topmargin 5mm \textheight 247mm

\newtheorem{theorem}{Theorem}
\newtheorem{lemma}{Lemma}

\theoremstyle{definition}
\newtheorem{exercise}{Exercise}

\begin{document}

\pagestyle{headings}
\noindent UNIVERSITE DE GENEVE \hfill Section de Math�matiques\\
\noindent Facult\'e des sciences \hfill \\[-3mm]
\hrule

\large

\begin{center}
\textbf{Topics In Probability And Analysis \\ Exercise Sheet 5 - Discussed on 22.10.2020}
\end{center}
\hrule
\text{}\\[1cm]

We define $\text{Slice}_t := \{(x,y): x-y = t\} \subset \mathbb{R}^2$.

\begin{exercise}
    Show that for two sets $A,B$ we have $\dim_M(A\times B \cap \text{Slice}_t) \leq (\dim_M(A \times B) - 1)_+$ for almost every $t$.
\end{exercise}

\begin{exercise}
    Review the construction for $\dim(A \times B) > \dim(A) + \dim(B)$.

    Show that if we only iterate the $A$ step, we get
    \[\dim(B) = 1, \; \dim(A) = \frac{1}{2}, \; \dim(A \times B) = \frac{3}{2}.\]

    Furthermore show that for $K_j = L_j = K \; \forall j$ we get 
    \[\dim(A) = \frac{3}{4} = \dim(B), \quad \dim(A \times B) = \frac{3}{4}.\]
\end{exercise}

\begin{exercise}
    Give explicit formulas for the contractions that have the Koch Snowflake and the Sierpinski Carpet as attractors.
\end{exercise}

\begin{exercise}
    Let $\text{CPT}(X) = \{K \subset X: K \text{ compact }, K\neq \emptyset\}$ and $d_H(K,L) = \inf \{\epsilon > 0: K \subset L_\epsilon, L \subset K_\epsilon\}$ the Hausdorff distance.

    Show that $d_H$ is a metric on $\text{CPT}(X)$.
\end{exercise}

\end{document}
