\documentclass[a4paper,11pt]{article}
\usepackage[french]{babel}
\usepackage[latin1]{inputenc}
%\usepackage{umlaut,amssymb,amsmath,amscd,a4,amsfonts}
\usepackage{amssymb,amsmath,amscd,a4,amsfonts,amsthm,mathrsfs}
%(a4 = 210 X 297 mm)
\hoffset -1in \voffset -1in \oddsidemargin 20mm \evensidemargin
\oddsidemargin \textwidth 170mm \topmargin 5mm \textheight 247mm

\newtheorem{theorem}{Theorem}
\newtheorem{lemma}{Lemma}

\theoremstyle{definition}
\newtheorem{exercise}{Exercise}

\begin{document}

\pagestyle{headings}
\noindent UNIVERSITE DE GENEVE \hfill Section de Math�matiques\\
\noindent Facult\'e des sciences \hfill \\[-3mm]
\hrule

\large

\begin{center}
\textbf{Topics In Probability And Analysis \\ Exercise Sheet 9 - Discussed on 02.12.2020}
\end{center}
\hrule
\text{}\\[1cm]

\begin{exercise}
	Let's consider the Whitney decomposition in one dimension. 
	\begin{enumerate}
		\item Let $\{Q_j\}$ be the Whitney decomposition of an interval $I$. Check that $\sum_j |Q_j|^s$ is comparable to $|I|^s$ for any $s \in (0,1]$.
		\item Conclude how one can compute the Whitney dimension for compact sets $K \subset \mathbb{R}$ without calculating the Whitney decomposition.
		\item Where does this go wrong in higher dimensions?
		\item Calculate the Whitney dimension of the Cantor set $K$ directly.
	\end{enumerate}
\end{exercise}

\begin{exercise}
	Write the Whitney dimension in terms of 
	\[\inf \{\beta: \; \int_{\text{outside of } K} \text{dist}(x,K)^\beta d\text{Vol}(x) < \infty\}.\]
\end{exercise}

\end{document}
