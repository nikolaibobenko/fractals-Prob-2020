\documentclass[a4paper,11pt]{article}
\usepackage[french]{babel}
\usepackage[latin1]{inputenc}
%\usepackage{umlaut,amssymb,amsmath,amscd,a4,amsfonts}
\usepackage{amssymb,amsmath,amscd,a4,amsfonts,amsthm,mathrsfs}
%(a4 = 210 X 297 mm)
\hoffset -1in \voffset -1in \oddsidemargin 20mm \evensidemargin
\oddsidemargin \textwidth 170mm \topmargin 5mm \textheight 247mm

\newtheorem{theorem}{Theorem}
\newtheorem{lemma}{Lemma}

\theoremstyle{definition}
\newtheorem{exercise}{Exercise}

\begin{document}

\pagestyle{headings}
\noindent UNIVERSITE DE GENEVE \hfill Section de Math�matiques\\
\noindent Facult\'e des sciences \hfill \\[-3mm]
\hrule

\large

\begin{center}
\textbf{Topics In Probability And Analysis \\ Exercise Sheet 7 - Discussed on 05.11.2020}
\end{center}
\hrule
\text{}\\[1cm]

We call a set $H \subset \mathbb{N}$ Poincar� if $\forall S \subset \mathbb{N}$ with $\bar{d}(S) > 0$ we have
\[(S-S) \cap H \neq \emptyset\]

\begin{exercise}
    Let $H \subset \mathbb{N}$ be finite. We denote by 
    \[ K_H = \{x = \sum x_n 2^{-n} :\; x_n x_{n+h} = 0 \text{ if } h \in H\}. \]
    Show that $\dim(K_H) = \dim_M(K_H)$. How can we calculate $\dim(K_H)$?
\end{exercise}

\begin{exercise}
    For the following sets check whether they are Poincar� or not:
    \begin{enumerate}
        \item $k \mathbb{N}$
        \item $2 \mathbb{N} + 1$
        \item $\{n^2\}_n$
        \item $\{n^2 + 1\}_n$
    \end{enumerate}
\end{exercise}



\end{document}
