\documentclass[a4paper,11pt]{article}
\usepackage[french]{babel}
\usepackage[latin1]{inputenc}
%\usepackage{umlaut,amssymb,amsmath,amscd,a4,amsfonts}
\usepackage{amssymb,amsmath,amscd,a4,amsfonts,amsthm,mathrsfs}
%(a4 = 210 X 297 mm)
\hoffset -1in \voffset -1in \oddsidemargin 20mm \evensidemargin
\oddsidemargin \textwidth 170mm \topmargin 5mm \textheight 247mm

\newtheorem{theorem}{Theorem}
\newtheorem{lemma}{Lemma}

\theoremstyle{definition}
\newtheorem{exercise}{Exercise}

\begin{document}

\pagestyle{headings}
\noindent UNIVERSITE DE GENEVE \hfill Section de Mathématiques\\
\noindent Facult\'e des sciences \hfill \\[-3mm]
\hrule

\large

\begin{center}
\textbf{Topics In Probability And Analysis \\ Exercise Sheet 8 - Discussed on 26.11.2020}
\end{center}
\hrule
\text{}\\[1cm]

\begin{exercise}
    We define the packing number as 
    \[N_p(K, \epsilon) = \max\{\# z_j \in K : \; |z_j - z_i| > \epsilon \; \forall i \neq  j\}.\]
    Show that
    \[N(K, 4\epsilon) \leq N_p(K, \epsilon) \leq N(K,\epsilon)\]
    and therefore the packing dimension and the Minkowski dimension agree.
\end{exercise}

\begin{exercise}
    Let $K\subset \mathbb{R}^d$. Denote by $K^\epsilon$ the $\epsilon$-neighborhood of $K$. Show that
    \[\overline{\dim}_M(K) = d + \overline{\lim_{\epsilon \to 0}} ;\ \frac{\log(\text{Vol}(K^\epsilon))}{\log(\frac{1}{\epsilon} )}.\]
\end{exercise}

\end{document}
