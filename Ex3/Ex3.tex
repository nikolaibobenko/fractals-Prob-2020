\documentclass[a4paper,11pt]{article}
\usepackage[french]{babel}
\usepackage[latin1]{inputenc}
%\usepackage{umlaut,amssymb,amsmath,amscd,a4,amsfonts}
\usepackage{amssymb,amsmath,amscd,a4,amsfonts,amsthm,mathrsfs}
%(a4 = 210 X 297 mm)
\hoffset -1in \voffset -1in \oddsidemargin 20mm \evensidemargin
\oddsidemargin \textwidth 170mm \topmargin 5mm \textheight 247mm

\newtheorem{theorem}{Theorem}
\newtheorem{lemma}{Lemma}

\theoremstyle{definition}
\newtheorem{exercise}{Exercise}

\begin{document}

\pagestyle{headings}
\noindent UNIVERSITE DE GENEVE \hfill Section de Math�matiques\\
\noindent Facult\'e des sciences \hfill \\[-3mm]
\hrule

\large

\begin{center}
\textbf{Topics In Probability And Analysis \\ Exercise Sheet 3 - Discussed on 08.10.2020}
\end{center}
\hrule
\text{}\\[1cm]

\begin{exercise}(Billingsley's Lemma with Vitali coverings)
	In class we showed Billingsley's Lemma for measures on the interval $[0,1]$. In the statement and proof we used the dyadic intervals $I_n(x)$. Show that we can generalize the statement by replacing this with a Vitali covering with the bounded subcover property.
\end{exercise}

\begin{exercise}
	We defined the dimension of a measure $\mu$ as
	\[\dim(\mu) = \inf\{\dim(A): \; A \text{ Borel}, \mu(A^c) = 0\}.\]

	Show that this is equivalent to
	\[\dim(\mu) = \inf\{\alpha: \mu \bot \mathscr{H}^\alpha\}\]
	where two measures $\mu, \nu$ are called orthogonal if there exists a set $A$ such that $\mu(A^c) = 0 = \nu(A)$.
\end{exercise}

\begin{exercise}
	Let $\mu$ be the measure assigning a mass $\frac{1}{2^n}$ to each $n$-th level interval of the middle-third Cantor set $K \subset [0,1]$. Calculate $\dim(\mu)$.
\end{exercise}

\begin{exercise}
	Construct a set $A$ which has zero Hausdorff measure of its own dimension which is not $0$. That is $\dim(A) = \alpha > 0$ and $\mathscr{H}^\alpha(A) = 0$.
\end{exercise}

\end{document}
